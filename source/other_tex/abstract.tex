\section*{Abstract}

We remove geometry as an input and show that a weak-field drift field
\(
  \mathbf{g} = -\nabla\Phi
\)
with
\(
  \Phi = \zeta\chi
\)
and
\(
  \nabla^{2}\Phi = 4\pi G\rho
\)
emerges from uniform, causal ticks of entropy--time.
Extending the S/t framework of the Planck-Cell program, gravity is treated
as a background drift in adjacency statistics, sourced by the same stiffness
parameter that defines inertial mass,
\(
  m = \hbar\omega_{0}/c^{2}.
\)
Defining
\(
  \chi = S/t
\)
and
\(
  \mathbf{g} = -\beta\nabla\chi,
\)
flux conservation for stationary sources yields
\(
  \nabla^{2}\Phi = 4\pi G\rho
\)
with
\(
  \Phi = \zeta\chi
\)
and
\(
  \mathbf{g} = -\nabla\Phi.
\)
This construction recovers Newton’s inverse-square law and its Gauss form
\(
  \nabla\!\cdot\!\mathbf{g} = -4\pi G\rho,
\)
while reproducing the weak-field gravitational redshift
\(
  \Delta f / f \approx \Delta\Phi / c^{2}.
\)
The equivalence principle arises naturally as a shared drift common to all
free bodies.
A worked Earth--Moon example gives
\(
  F \sim 2\times10^{20}\,\mathrm{N}
\)
in the Newtonian limit using standard planetary parameters.
