\section*{Equivalence principle (drift version)}
Why do heavy and light objects fall together?
\begin{itemize}
  \item Inertia: more stiffness $\Rightarrow$ more resistance to pushes.
  \item Gravity: is not an applied force. It is a background drift. The adjacency network leans by the same amount each tick for all lumps.
\end{itemize}
The same knob ($\omega_0$) that resists applied acceleration also sources the
local drift that other geodesics follow. Since the inertial response scales with
$\omega_0$ and the source strength of the bias likewise scales with $\omega_0$,
we have $m_{\text{inertial}} = m_{\text{grav}}$ by construction; hence the
free-fall acceleration is common: $\boldsymbol{a} = \boldsymbol{g}$ for all
test bodies. This dual role is consistent with the GR formulation and precision
tests~\cite{einstein1916foundation,will2014confrontation}.


\section*{From flux to Newton’s law}
Recall the identifications
\[
\chi \;=\; \frac{S}{t},\qquad \mathbf g\ :=\ -\,\beta\,\nabla\chi,\qquad \mathbf F \;=\; m\,\mathbf g.
\]
\emph{Flux postulate (stationary sources).} For a stationary, spherically symmetric source embedded in an asymptotically uniform Planck-Cell background, the outward $\chi$-flux through spheres $S_r$ is radius-independent (no sinks/sources between spheres). With $A=4\pi r^2$,
\[
\oint_{S_r}\nabla\chi\cdot d\mathbf A \;=\; C\,M_{\rm enclosed}
\quad\Longleftrightarrow\quad
\nabla^2\chi \;=\; C\,\rho.
\]
Define the Newtonian potential by
\[
\Phi \;:=\; \beta\,\chi \quad\Rightarrow\quad \mathbf g=-\,\nabla\Phi,
\]
so that
\[
\nabla^2\Phi \;=\; \beta\,\nabla^2\chi \;=\; \beta C\,\rho \;=\; 4\pi G\,\rho,
\]
with the identification \(G := \beta C/(4\pi)\).
Only $\beta C$ is fixed in this limit, consistent with the normalization freedom discussed in \cite{langstaff2025_temporal_relativity_entropy_clock}.

\paragraph{Spherical solution and superposition (weak field).}
Spherical symmetry yields
\[
\nabla\Phi \;=\; \frac{G M}{r^2}\,\hat{\mathbf r}
\quad\Rightarrow\quad
\mathbf g \;=\; -\,\frac{G M}{r^2}\,\hat{\mathbf r}.
\]
In this linear (weak-field) regime, superposition holds. Strong gradients require the curved-adjacency (nonlinear) treatment developed later, paralleling the transition from Newtonian gravity to GR \cite{einstein1916foundation}.

\begin{tcolorbox}
\textbf{Units map (so that $\mathbf g$ is in $\mathrm{m\,s^{-2}}$).}\\
$S$ is dimensionless (nats), so $\chi=S/t$ has units $\mathrm{s^{-1}}$ and $\nabla\chi$ has units $\mathrm{s^{-1}\,m^{-1}}$.\\
Choose $[\beta]=\mathrm{m^2\,s^{-1}}$ and $[C]=\mathrm{s^{-1}\,m\,kg^{-1}}$ so that
$[\beta C]=\mathrm{m^3\,kg^{-1}\,s^{-2}}=[G]$, giving $\mathbf g=-\beta\nabla\chi$ in $\mathrm{m\,s^{-2}}$.
\end{tcolorbox}

\paragraph{Gauss form and normalization.}
Since $\mathbf g=-\nabla\Phi$ and $\nabla^2\Phi=4\pi G\,\rho$, one has the equivalent Gauss law
\[
\nabla\!\cdot\!\mathbf g \;=\; -\,4\pi G\,\rho.
\]

\paragraph{Clock redshift (weak field).}
Clock rates tilt with $\chi$: for two stationary heights,
\[
\frac{\Delta f}{f} \;\approx\; \frac{\Delta \Phi}{c^2}\,,
\]
recovering the gravitational redshift at leading order \cite{einstein1916foundation,will2014confrontation}.

\paragraph{Tides (geodesic-deviation teaser).}
Define the tidal tensor $\mathsf{T}_{ij}:=\partial_i\partial_j\Phi$; nearby free-fall worldlines obey
\[
\delta\ddot{x}_i \;\approx\; -\,\mathsf{T}_{ij}\,\delta x_j,
\]
foreshadowing curvature and the relativistic treatment \cite{einstein1916foundation}.

\clearpage
\section*{Worked Example: Earth--Moon Attraction}
Using CODATA’s value of $G$ \cite{codata2018}, Earth and Moon parameters from NASA fact sheets
\cite{nasa_earth_fact,nasa_moon_fact}, and mean Earth–Moon distance \cite{nasa_moon_fact}:

\[
G \approx 6.674\times 10^{-11}\ \mathrm{m^3\,kg^{-1}\,s^{-2}}
\]
\[
m_E\approx 5.97\times 10^{24}\ \mathrm{kg}
\]
\[
m_M\approx 7.35\times 10^{22}\ \mathrm{kg}
\]
\[
r\approx 3.84\times 10^{8}\ \mathrm{m}
\]

Newton’s law gives
\[
F = G\frac{m_E m_M}{r^2}
   \approx 2.0\times 10^{20}\ \mathrm{N}.
\]

\begin{tcolorbox}
\textbf{Example (Earth--Moon system).}\\
The mutual stiffness bias between Earth and Moon produces a gravitational pull of order
$10^{20}$ newtons. In the Planck-Cell picture, it is the time--entropy flux gently reweighting
adjacency between two stiffness clocks over $r$ hops, keeping the Moon in orbit.
\end{tcolorbox}

\section*{Beyond Newton (foreshadowing)}
When gradients in $\chi$ within the Planck-Cell medium are strong (near horizons), the $1/r^2$ scaling is modified.
The full picture is curved adjacency: geodesics bend as if space itself is warped.
Benchmark requirements for the curved-adjacency model include:
\begin{itemize}
  \item light deflection with the GR factor-of-two relative to Newtonian bending \cite{eddington1920,will2014confrontation},
  \item Shapiro time delay \cite{will2014confrontation},
  \item perihelion precession of bound orbits \cite{will2014confrontation},
  \item consistency with gravitational redshift already captured at leading order \cite{einstein1916foundation,will2014confrontation}.
\end{itemize}

\paragraph{Universality test hook.}
Define the E\"otv\"os parameter for two test bodies 1,2:
\[
\eta \;=\; \frac{2\,|a_1-a_2|}{a_1+a_2}\,.
\]
In this framework $\eta=0$ at leading order. The MICROSCOPE mission reports $|\eta|\lesssim 2\times 10^{-15}$ \cite{touboul2022microscope}, consistent with universality in the weak-field regime.

%%% PATCH BEGIN: KM-PL-001
\section*{Clarifications}
\paragraph{Grain scale vs Planck length}
\(\xi_g\) denotes the effective adjacency grain; we require \(\lambda\gg\xi_g\) for continuum behavior. Operationally \(\xi_g\ge \ell_P\).
\paragraph{``Unitary'' means information-preserving}
Local updates are bijective maps on neighborhood state; no Hilbert-space structure is assumed here.
\paragraph{Lorentz recovery (sketch)}
Assume hop-cap speed \(c\) and reciprocity. Preserving null lines \(x=\pm ct\) under a linear inertial change yields
\(x'=\gamma(x-vt),\ t'=\gamma(t-\frac{v}{c^2}x)\) with \(\gamma=(1-v^2/c^2)^{-1/2}\) \cite{einstein1905}.
\paragraph{Long-wavelength residual law}
For \(\lambda\gg\xi_g\), fractional residuals satisfy \(r(\lambda)\lesssim A(\xi_g/\lambda)^p\) with \(p>0\).
\paragraph{Shared front speed}
If EM and gravity obey the same hop cap, their fronts share \(c\); multi-messenger data are consistent \cite{abbott2017gw170817}.
%%% PATCH END: KM-PL-001


\paragraph{Lemma (flux $\to$ Poisson).}
Assume a stationary source with density $\rho$ and a $\chi$-field whose flux through any sphere $S_r$ enclosing mass $M(r)$ is proportional to $M(r)$:
\[\oint_{S_r} \nabla\chi\cdot dA = C\,M(r).\]
By the divergence theorem this implies $\nabla^2\chi = C\,\rho$. With $\Phi:=\zeta\,\chi$ we obtain Poisson's equation $\nabla^2\Phi=4\pi G\rho$ with $G=\zeta C/(4\pi)$, and hence $g=-\nabla\Phi$.

\paragraph{Units.}
\begin{description}
\item[$\chi$] s$^{-1}$ (entropy--time rate)
\item[$\Phi$] m$^2$ s$^{-2}$ (potential)
\item[$g$] m s$^{-2}$ (drift/acceleration)
\item[$G$] m$^3$ kg$^{-1}$ s$^{-2}$ with $G=\zeta C/(4\pi)$
\end{description}

\paragraph{Conclusion.}
From $\Delta S=\Delta\tau$ and one--hop--per--tick exchange, conservation of $\chi$-flux yields the weak-field Newton law via $\Phi=\zeta\,\chi$ and $g=-\nabla\Phi$. This makes the equivalence principle immediate (shared drift), with normalization fixed by $G=\zeta C/(4\pi)$. The next step is to treat strong-field regimes where adjacency itself curves (curved-adjacency / GR limit).