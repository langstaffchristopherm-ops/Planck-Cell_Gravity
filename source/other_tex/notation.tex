\section*{Notation}

\noindent\textbf{Conventions.}  
Time and entropy advance in discrete, \emph{dimensionless ticks}.   
A single primitive cell unit is created per tick, giving
\[
  \Delta S = \Delta \tau.
\]
Physical mapping is applied only when necessary:
\[
  S_{\mathrm{phys}} = k_B\,S, \qquad
  \tau_s = t_P\,\tau.
\]
Fundamental constants: speed of light \(c\), reduced Planck constant \(\hbar\), and Boltzmann constant \(k_B\).

\smallskip
\noindent\textit{Native--SI correspondence.}  
One hop per tick implies \(\mathrm{d}a/\mathrm{d}\tau = 1\).  
Choosing a length per hop \(\ell_0\) and a time per tick \(t_0\) yields
\[
  c = \frac{\ell_0}{t_0}.
\]

\bigskip
\noindent\textbf{Core quantities.}
\begin{description}[leftmargin=2.4em,labelsep=0.8em]
  \item[\(k\)] Tick index (integer). The scale factor \(a(k)\) is defined with respect to ticks; baseline \(a(k)\propto k^{1/3}\).
  \item[\(\Delta S,\,\Delta\tau\)] Primitive-cell and proper-time tick \emph{counts} (dimensionless) satisfying \(\Delta S=\Delta\tau\).
  \item[\(\chi\)] Local entropy–time rate, \(\chi := S/t\) (s\(^{-1}\)).
  \item[\(\zeta\)] Mapping constant linking \(\chi\) to the potential \(\Phi\) (m\(^2\) s\(^{-1}\)).
  \item[\(g\)] Drift field defined by \(g:=-\nabla\Phi\) (m s\(^{-2}\)).
  \item[\(C\)] Source constant in the \(\chi\)-flux law (s\(^{-1}\) m kg\(^{-1}\)).
  \item[\(G\)] Newton’s constant with \(G = \zeta C / (4\pi)\) (m\(^3\) kg\(^{-1}\) s\(^{-2}\)).
\end{description}

\noindent\textbf{Graph and geometric parameters.}
\begin{description}[leftmargin=2.4em,labelsep=0.8em]
  \item[\(d_G(x,y)\)] Graph (hop) distance between vertices \(x\) and \(y\).
  \item[\(\ell_N\)] Typical interior hop length for a sample of size \(N\).
  \item[\(\varepsilon_N,\,\delta_N\)] Mesh non-uniformity and directional-bias parameters (\(\to 0\) as \(N\to\infty\)).
  \item[\(\eta_N\)] Combined small parameter: \(\eta_N := C_1\varepsilon_N + C_2\delta_N.\)
  \item[\(\mathrm{distortion}(N)\)] Bilipschitz distortion of the graph metric relative to the ambient metric:
  \[
    \mathrm{distortion}(N)
      = \sup_{x,y}
        \max\!\left\{
          \frac{d_G(x,y)}{\|x-y\|},\,
          \frac{\|x-y\|}{d_G(x,y)}
        \right\}-1.
  \]
\end{description}

\noindent\textbf{Gravity-specific quantities.}
\begin{description}[leftmargin=2.4em,labelsep=0.8em]
  \item[\(\Phi\)] Newtonian potential; \(\mathbf{g} = -\nabla\Phi\) gives the local gravitational acceleration.
  \item[\(M,\,r\)] Source mass and radial coordinate (weak-field limit).
  \item[\(\beta\)] Empirical slope in \(\Delta\nu/\nu = \beta\,\Delta\Phi / c^2\) (expected \(\beta \simeq 1\)).
  \item[\(\chi\)] Local entropy–time rate, \(\chi := S/t\) (s\(^{-1}\)).
  \item[\(\zeta\)] Mapping constant between \(\chi\) and \(\Phi\) (m\(^2\) s\(^{-1}\)).
  \item[\(g\)] Drift field \(g:=-\nabla\Phi\) (m s\(^{-2}\)).
  \item[\(C\)] Source constant in the \(\chi\)-flux law (s\(^{-1}\) m kg\(^{-1}\)).
  \item[\(G\)] Newton’s constant \(G = \zeta C / (4\pi)\) (m\(^3\) kg\(^{-1}\) s\(^{-2}\)).
\end{description}

\noindent Gradients of the potential define the local acceleration \(\mathbf{g} = \nabla\Phi\).  
Elsewhere, the explicit rate form \(\dot{\mathcal{S}} = \mathrm{d}S/\mathrm{d}t\) is used for clarity.  
The two notions are related but distinct.

\begin{tcolorbox}[colback=gray!5!white,colframe=black!15!black,title=\textbf{Key distinction}]
\begin{itemize}[leftmargin=1em]
  \item \textbf{Inertia:} stiffness resists \emph{applied pushes}.  
  \item \textbf{Gravity:} not an applied force—it is a \emph{background drift}.  
  The Planck-Cell fabric tilts, so all free objects drift together regardless of stiffness.
\end{itemize}
\end{tcolorbox}

\bigskip
\noindent\textbf{Quick symbol summary.}
\begin{description}[leftmargin=2.4em,labelsep=0.8em]
  \item[\(\ell_P\)] Planck-cell edge or hop length.
  \item[\(t_P\)] Planck tick (one global update interval).
  \item[\(c=\ell_P/t_P\)] Causal speed cap (one hop per tick).
  \item[\(k\)] Discrete tick index (global ledger step).
  \item[\(R(k)\)] Front radius after \(k\) ticks.
  \item[\(N(k)\)] Number of active/born cells after \(k\) ticks.
  \item[\(L\)] Continuum baseline distance.
  \item[\(d_G\)] Graph (hop) distance.
  \item[\(T(L)\)] Transit time over baseline \(L\).
  \item[\(v_g\)] Group or front speed.
  \item[\(E,\,\hbar\)] Energy and reduced Planck constant.
  \item[\(\Delta\phi\)] Accumulated phase along the path.
  \item[\(\lambda\)] Wavelength of the probe signal.
\end{description}
