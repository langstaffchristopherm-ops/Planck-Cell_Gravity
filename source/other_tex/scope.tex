\section*{Scope and falsifiability}
This paper extends the \emph{Planck-Cell S/t framework} into the weak-field
gravitational regime, modelling gravity as a background drift of the
entropy–time rate~$\chi=S/t$ rather than as an applied force.
All claims apply to interior regions of a bounded-degree, approximately
isotropic adjacency network in which no entity advances more than one hop
per global tick.  The construction is falsified by any measurable breakdown
of the drift correspondence between inertia and gravity or by departures
from the derived flux law
\[
\nabla^{2}\Phi=4\pi G\rho,\qquad g=-\nabla\Phi,\qquad G=\zeta C/(4\pi).
\]
Immediate falsifiers include:
\begin{enumerate}[label=(\roman*)]
  \item a reproducible deviation of the gravitational-redshift slope
        $\hat{\beta}$ from unity in $\Delta\nu/\nu=\hat{\beta}\,\Delta\Phi/c^{2}$;
  \item a measurable violation of the universality of free fall,
        quantified by an Eötvös parameter $|\eta|>2\times10^{-15}$;
  \item failure of the inferred Newton-law normalization
        $G=\zeta C/(4\pi)$ to remain constant across source systems;
  \item persistent discrepancy between the predicted and observed $1/r^{2}$
        force law or tidal tensor in the weak-field limit.
\end{enumerate}
Any confirmed non-zero~$\eta$, anomalous~$\hat{\beta}$, or radius-dependent
drift flux would falsify the gravitational layer of the framework.