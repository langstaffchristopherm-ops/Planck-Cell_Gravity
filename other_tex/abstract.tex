% other_tex/abstract.tex

\noindent \textbf{Planck-Cell Dynamics — Gravity in the Planck-Cell Medium.}

We extend the $S/t$ framework of the Planck-Cell program by treating gravity as a background
\emph{local drift} of adjacency statistics, sourced by the same stiffness knob that defines inertial
mass ($m=\hbar\wzero/c^2$) \cite{langstaff2025_temporal_relativity_entropy_clock,langstaff2025_planck_cell_kinematics,langstaff2025_planck_cell_mass}.

Defining $\chi=S/t$ and $\mathbf g=-\,\beta\nabla\chi$, flux conservation for stationary sources yields
$\nabla^2\Phi=4\pi G\,\rho$ with $\Phi=\beta\chi$ and $\mathbf g=-\nabla\Phi$. This construction recovers Newton’s
inverse-square law and its Gauss form $\nabla\!\cdot\!\mathbf g=-4\pi G\,\rho$ \cite{newton1687principia}.

The framework also reproduces the weak-field gravitational redshift
$\Delta f/f\approx\Delta\Phi/c^2$ \cite{einstein1916foundation,will2014confrontation}, while keeping the equivalence
principle explicit as a shared drift common to all free bodies.

Finally, a worked Earth–Moon example gives
$F\!\sim\!2\times10^{20}\,\mathrm N$ in the Newtonian limit, using $G$ and planetary parameters from
standard references \cite{codata2018,nasa_earth_fact,nasa_moon_fact}.
