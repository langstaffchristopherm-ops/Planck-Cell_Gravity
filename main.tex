\documentclass[11pt,oneside]{article}

% All packages/macros live here:
\input{other_tex/format_macros.tex}

% Optional math + theorem layers (safe if absent/empty)
\IfFileExists{other_tex/math.tex}{% Local guards
\providecommand{\wzero}{\omega_0}
\providecommand{\biascouple}{\beta}
\providecommand{\Sent}{\mathcal S}
\providecommand{\chiRate}{\dot{\Sent}}}{}
\IfFileExists{other_tex/theorems.tex}{\input{other_tex/theorems.tex}}{}

% -------------
% Metadata
% -------------
\title{\PaperTitleMain}
\ifdefined\Authors
  \author{\Authors}
\else\ifdefined\AuthorName
  \author{\AuthorName}
\else
  \author{Author Name}
\fi\fi
\date{\today}

\begin{document}
\maketitle

% Dedication inline under the title (no page break)
\IfFileExists{other_tex/dedication.tex}{%
  \vspace{0.75\baselineskip}\cleardoublepage
\begingroup
  \hypersetup{pageanchor=false}% prevent duplicate 'page.i' anchor
  \thispagestyle{empty}
  \vspace*{\fill}
  \begin{center}\itshape
  To those who think themselves alone—may you find friends who turn doubts into spells, and open doors when your heart requires them most.
  \end{center}
  \vspace*{\fill}
  \clearpage
\endgroup%
}{}

% Abstract on its own page
\clearpage
\section*{Abstract}
\addcontentsline{toc}{section}{Abstract}
\section*{Abstract}
We remove geometry as an input and show that a weak-field drift field $g=-\nabla\Phi$ with $\Phi=\zeta\,\chi$ and $\nabla^2\Phi=4\pi G\rho$ emerges from uniform, causal ticks of entropy--time.
% other_tex/abstract.tex

\noindent \textbf{Planck-Cell Dynamics — Gravity in the Planck-Cell Medium.}

We extend the $S/t$ framework of the Planck-Cell program by treating gravity as a background
\emph{local drift} of adjacency statistics, sourced by the same stiffness knob that defines inertial
$m = \hbar \omega_0 / c^2$ \cite{langstaff2025_temporal_relativity_entropy_clock,langstaff2025_planck_cell_kinematics,langstaff2025_planck_cell_mass}.


Defining $\chi=S/t$ and $\mathbf g=-\,\beta\nabla\chi$, flux conservation for stationary sources yields
$\nabla^2\Phi=4\pi G\,\rho$ with $\Phi=\zeta\chi$ and $\mathbf g=-\nabla\Phi$. This construction recovers Newton’s
inverse-square law and its Gauss form $\nabla\!\cdot\!\mathbf g=-4\pi G\,\rho$ \cite{newton1687principia}.

The framework also reproduces the weak-field gravitational redshift
$\Delta f/f\approx\Delta\Phi/c^2$ \cite{einstein1916foundation,will2014confrontation}, while keeping the equivalence
principle explicit as a shared drift common to all free bodies.

Finally, a worked Earth–Moon example gives
$F\!\sim\!2\times10^{20}\,\mathrm N$ in the Newtonian limit, using $G$ and planetary parameters from
standard references \cite{codata2018,nasa_earth_fact,nasa_moon_fact}.
% NOTE: Ensure the Abstract refers to the “Planck-Cell medium” (not “Entropy medium”)
% and may optionally cite \cite{langstaff2025_temporal_relativity_entropy_clock,langstaff2025_planck_cell_kinematics,langstaff2025_planck_cell_mass}.

% ================== Gravity in the Planck-Cell Medium (Article Body) ==================

% ---- Local safety shims (do not require changing preamble) ----
\makeatletter
\@ifundefined{tcolorbox}{\newenvironment{tcolorbox}{}{} }{} % fallback if tcolorbox is not loaded
\makeatother

% ---- Local guards (safe if already defined elsewhere) ----
\providecommand{\wzero}{\omega_0}
\providecommand{\biascouple}{\beta}
\providecommand{\Sent}{\mathcal S}
\providecommand{\chiRate}{\dot{\Sent}}

\section*{Notation}

\begin{table}[!htbp]
\centering
\small
\renewcommand{\arraystretch}{1.2}
\begin{tabular}{@{} l l >{\raggedright\arraybackslash}p{0.58\linewidth} @{}}
\hline
\textbf{Symbol} & \textbf{Units} & \textbf{Meaning / Notes} \\
\hline
$\tau$ & s & Fundamental tick interval (time increment per update of the Planck-Cell medium). \\
$S$ & nat & Dimensionless entropy (nats); defines $\chi:=S/t$. \\
$\chi(\mathbf x)$ & s$^{-1}$ & Entropy-per-time potential; its gradient biases local hops. \\
$\beta$ & m$^{2}$\,s$^{-1}$ & Coupling converting $\nabla\chi$ into drift; $\mathbf g=-\,\beta\nabla\chi$. Units chosen so that $\mathbf g$ is in m\,s$^{-2}$. \\
$C$ & s$^{-1}$\,m\,kg$^{-1}$ & Proportionality linking $\nabla^2\chi$ to mass density $\rho$ (stationary, spherically symmetric regime). \\
$\Phi$ & m$^{2}$\,s$^{-2}$ & Newtonian potential $\Phi=\beta\,\chi$ (defined up to an additive constant); $\mathbf g=-\nabla\Phi$. \\
$\rho$ & kg\,m$^{-3}$ & Mass density. \\
$M$ & kg & Enclosed mass. \\
$r,\ \hat{\mathbf r}$ & m,\ — & Radial distance $r=\|\mathbf x\|$; unit vector $\hat{\mathbf r}=\mathbf x/r$. \\
$\wzero$ & s$^{-1}$ & Intrinsic clock stiffness (angular-frequency scale) giving $m=\hbar\wzero/c^2$ \cite{langstaff2025_planck_cell_mass}. \\
\hline
\end{tabular}
\end{table}

\noindent\emph{Identifiability note.} Only the product $\beta\,C$ is observable in the Newtonian limit. Individual values of $\beta$ and $C$ require a convention (e.g., absorb $\beta$ into $\chi$), consistent with the normalization choices in \cite{langstaff2025_temporal_relativity_entropy_clock}.


\noindent\emph{Identifiability note.} Only the product $\beta\,C$ is observable in the Newtonian limit. Individual values of $\beta$ and $C$ require a convention (e.g., absorb $\beta$ into $\chi$), consistent with the normalization choices in \cite{langstaff2025_temporal_relativity_entropy_clock}.


\paragraph{Assumptions (regime for this article).}
Weak fields, slow motion, quasi-static sources, and locally homogeneous patches away from boundaries (Newtonian limit; cf.\ weak-field limit of GR \cite{einstein1916foundation,will2014confrontation}).

\paragraph{Sign and boundary conventions.}
We fix $\Phi(\infty)=0$ for isolated stationary sources, so $\Phi(r)=-GM/r$ for a point source. Field lines are inward: $\mathbf g=-\nabla\Phi$.

\section*{Scope}
Gravity appears as the canonical case of constant acceleration (magnitude $a=g$), historically captured by Newton’s inverse-square law \cite{newton1687principia}. Here we deepen the picture:
\begin{enumerate}
  \item Inertia $\Leftrightarrow$ temporal stiffness: resistance to changing tick--hop ratios (kinematics layer in \cite{langstaff2025_planck_cell_kinematics}).
  \item Gravity $\Leftrightarrow$ fabric bias: a \emph{local drift} of adjacency statistics each tick in the Planck-Cell medium (framework initiated in \cite{langstaff2025_temporal_relativity_entropy_clock}).
  \item The equivalence principle emerges automatically: the same stiffness knob governs both inertia and gravitational sourcing, consistent with experimental tests \cite{will2014confrontation}.
\end{enumerate}
We derive Newton’s inverse-square law from time--entropy flux conservation within the Planck-Cell medium, fix normalizations and boundary conditions, and check numerically for the Earth--Moon system.

\section{Recap: inertia from stiffness}
We define mass as temporal stiffness,
\[
m=\frac{\hbar\wzero}{c^2},
\]
as developed in \cite{langstaff2025_planck_cell_mass}. A larger $\wzero$ means stronger resistance to rephasing—more ``cost'' to change the hop ratio per tick. This explains why massive objects accelerate less under the same push: they are stiffer clocks.

\section{Gravity as local drift (adjacency bias)}
\textbf{Goal.} Construct a local drift field $\mathbf g(\mathbf x)$ common to all free bodies and yielding an inverse-square law via flux conservation.

Within the Planck-Cell medium, each tick can include a small systematic bias of adjacency updates. Instead of ``straight'' updates, the hop distribution \emph{leans}. We model a \emph{local drift field}
\[
\mathbf g(\mathbf x)\ :=\ -\,\beta\,\nabla\chi(\mathbf x),\qquad
\Delta\mathbf v \;=\; \mathbf g\,\Delta t\quad(\Delta t=\tau\ \text{per tick}),
\]
where $\chi=S/t$ and $\beta$ is fixed by matching to Newton’s law. In the hop picture, gravity is a \emph{reweighting} of paths: all free trajectories acquire the same local drift over small regions (while $\mathbf g(\mathbf x)$ may vary slowly with position). This drift formalism complements the kinematic rules in \cite{langstaff2025_planck_cell_kinematics}.

\paragraph{Terminology.}
Here $\chi$ is an \emph{entropy-per-time potential} (units $\,\mathrm{s^{-1}}$) whose gradients define $\mathbf g$. Elsewhere we use $\dot{\mathcal S}=dS/dt$ for explicit rates. The two notions are related but distinct.

\begin{tcolorbox}
\textbf{Key distinction.}\\
-- Inertia: stiffness resists \emph{applied pushes}.\\
-- Gravity: not an applied force; it is a \emph{background drift}. The Planck-Cell fabric tilts, so all free objects drift together regardless of stiffness.
\end{tcolorbox}

\section{Equivalence principle (drift version)}
Why do heavy and light objects fall together?
\begin{itemize}
  \item Inertia: more stiffness $\Rightarrow$ more resistance to pushes.
  \item Gravity: is not an applied force. It is a background drift. The adjacency network leans by the same amount each tick for all lumps.
\end{itemize}
The same knob ($\wzero$) that resists applied acceleration also sources the local drift that other geodesics follow. Since the inertial response scales with $\wzero$ and the source strength of the bias likewise scales with $\wzero$, we have $m_{\text{inertial}}=m_{\text{grav}}$ by construction; hence the free-fall acceleration is common: $\mathbf a=\mathbf g$ for all test bodies. This dual role is consistent with the GR formulation and precision tests \cite{einstein1916foundation,will2014confrontation}.

\section{From flux to Newton’s law}
Recall the identifications
\[
\chi \;=\; \frac{S}{t},\qquad \mathbf g\ :=\ -\,\beta\,\nabla\chi,\qquad \mathbf F \;=\; m\,\mathbf g.
\]
\emph{Flux postulate (stationary sources).} For a stationary, spherically symmetric source embedded in an asymptotically uniform Planck-Cell background, the outward $\chi$-flux through spheres $S_r$ is radius-independent (no sinks/sources between spheres). With $A=4\pi r^2$,
\[
\oint_{S_r}\nabla\chi\cdot d\mathbf A \;=\; C\,M_{\rm enclosed}
\quad\Longleftrightarrow\quad
\nabla^2\chi \;=\; C\,\rho.
\]
Define the Newtonian potential by
\[
\Phi \;:=\; \beta\,\chi \quad\Rightarrow\quad \mathbf g=-\,\nabla\Phi,
\]
so that
\[
\nabla^2\Phi \;=\; \beta\,\nabla^2\chi \;=\; \beta C\,\rho \;=\; 4\pi G\,\rho,
\]
with the identification \(G := \beta C/(4\pi)\).
Only $\beta C$ is fixed in this limit, consistent with the normalization freedom discussed in \cite{langstaff2025_temporal_relativity_entropy_clock}.

\paragraph{Spherical solution and superposition (weak field).}
Spherical symmetry yields
\[
\nabla\Phi \;=\; \frac{G M}{r^2}\,\hat{\mathbf r}
\quad\Rightarrow\quad
\mathbf g \;=\; -\,\frac{G M}{r^2}\,\hat{\mathbf r}.
\]
In this linear (weak-field) regime, superposition holds. Strong gradients require the curved-adjacency (nonlinear) treatment developed later, paralleling the transition from Newtonian gravity to GR \cite{einstein1916foundation}.

\begin{tcolorbox}
\textbf{Units map (so that $\mathbf g$ is in $\mathrm{m\,s^{-2}}$).}\\
$S$ is dimensionless (nats), so $\chi=S/t$ has units $\mathrm{s^{-1}}$ and $\nabla\chi$ has units $\mathrm{s^{-1}\,m^{-1}}$.\\
Choose $[\beta]=\mathrm{m^2\,s^{-1}}$ and $[C]=\mathrm{s^{-1}\,m\,kg^{-1}}$ so that
$[\beta C]=\mathrm{m^3\,kg^{-1}\,s^{-2}}=[G]$, giving $\mathbf g=-\beta\nabla\chi$ in $\mathrm{m\,s^{-2}}$.
\end{tcolorbox}

\paragraph{Gauss form and normalization.}
Since $\mathbf g=-\nabla\Phi$ and $\nabla^2\Phi=4\pi G\,\rho$, one has the equivalent Gauss law
\[
\nabla\!\cdot\!\mathbf g \;=\; -\,4\pi G\,\rho.
\]

\paragraph{Clock redshift (weak field).}
Clock rates tilt with $\chi$: for two stationary heights,
\[
\frac{\Delta f}{f} \;\approx\; \frac{\Delta \Phi}{c^2}\,,
\]
recovering the gravitational redshift at leading order \cite{einstein1916foundation,will2014confrontation}.

\paragraph{Tides (geodesic-deviation teaser).}
Define the tidal tensor $\mathsf{T}_{ij}:=\partial_i\partial_j\Phi$; nearby free-fall worldlines obey
\[
\delta\ddot{x}_i \;\approx\; -\,\mathsf{T}_{ij}\,\delta x_j,
\]
foreshadowing curvature and the relativistic treatment \cite{einstein1916foundation}.

\section{Worked Example: Earth--Moon Attraction}
Using CODATA’s value of $G$ \cite{codata2018}, Earth and Moon parameters from NASA fact sheets
\cite{nasa_earth_fact,nasa_moon_fact}, and mean Earth–Moon distance \cite{nasa_moon_fact}:

\[
G \approx 6.674\times 10^{-11}\ \mathrm{m^3\,kg^{-1}\,s^{-2}}
\]
\[
m_E\approx 5.97\times 10^{24}\ \mathrm{kg}
\]
\[
m_M\approx 7.35\times 10^{22}\ \mathrm{kg}
\]
\[
r\approx 3.84\times 10^{8}\ \mathrm{m}
\]

Newton’s law gives
\[
F = G\frac{m_E m_M}{r^2}
   \approx 2.0\times 10^{20}\ \mathrm{N}.
\]

\begin{tcolorbox}
\textbf{Example (Earth--Moon system).}\\
The mutual stiffness bias between Earth and Moon produces a gravitational pull of order
$10^{20}$ newtons. In the Planck-Cell picture, it is the time--entropy flux gently reweighting
adjacency between two stiffness clocks over $r$ hops, keeping the Moon in orbit.
\end{tcolorbox}

\section{Beyond Newton (foreshadowing)}
When gradients in $\chi$ within the Planck-Cell medium are strong (near horizons), the $1/r^2$ scaling is modified.
The full picture is curved adjacency: geodesics bend as if space itself is warped.
Benchmark requirements for the curved-adjacency model include:
\begin{itemize}
  \item light deflection with the GR factor-of-two relative to Newtonian bending \cite{eddington1920,will2014confrontation},
  \item Shapiro time delay \cite{will2014confrontation},
  \item perihelion precession of bound orbits \cite{will2014confrontation},
  \item consistency with gravitational redshift already captured at leading order \cite{einstein1916foundation,will2014confrontation}.
\end{itemize}

\paragraph{Universality test hook.}
Define the E\"otv\"os parameter for two test bodies 1,2:
\[
\eta \;=\; \frac{2\,|a_1-a_2|}{a_1+a_2}\,.
\]
In this framework $\eta=0$ at leading order. The MICROSCOPE mission reports $|\eta|\lesssim 2\times 10^{-15}$ \cite{touboul2022microscope}, consistent with universality in the weak-field regime.

% ----------------
% Acknowledgements (requested to appear before Competing interests)
% ----------------
\clearpage
\section*{Acknowledgements}
\addcontentsline{toc}{section}{Acknowledgements}
\IfFileExists{other_tex/acknowledgments.tex}{\input{other_tex/acknowledgments.tex}}{%
The author thanks colleagues and reviewers for comments on the Planck-Cell framework and its gravitational extension, and acknowledges helpful discussions related to prior works \cite{langstaff2025_temporal_relativity_entropy_clock,langstaff2025_planck_cell_kinematics,langstaff2025_planck_cell_mass}.
}

% ----------------
% Statements & housekeeping
% ----------------
\section*{Competing interests}
\IfFileExists{other_tex/competing_interests.tex}{\input{other_tex/competing_interests.tex}}{}

\section*{Author contributions}
\IfFileExists{other_tex/authors_contributions.tex}{\input{other_tex/authors_contributions.tex}}{}

\section*{Funding}
\IfFileExists{other_tex/funding.tex}{\input{other_tex/funding.tex}}{}

\section*{Data and materials availability}
\IfFileExists{data_accessibility.tex}{\input{data_accessibility.tex}}{No new data were generated.}

\section*{Disclosure}
\IfFileExists{other_tex/disclosure.tex}{\input{other_tex/disclosure.tex}}{}

% ==================
% Other Works Section
% ==================
\clearpage
\section*{Other Works by the Author}
\addcontentsline{toc}{section}{Other Works by the Author}

The present article on gravity in the Planck-Cell medium builds on a sequence of related works that establish the framework step by step:

\begin{itemize}
  \item \textbf{Temporal Relativity.}  
  C.~M. Langstaff (2025). Zenodo.  
  DOI: \href{https://doi.org/10.5281/zenodo.17119049}{10.5281/zenodo.17119049}.
  
  \item \textbf{Planck-Cell Kinematics.}  
  C.~M. Langstaff (2025). Zenodo.  
  DOI: \href{https://doi.org/10.5281/zenodo.17168478}{10.5281/zenodo.17168478}.
  
  \item \textbf{Planck-Cell Mass.}  
  C.~M. Langstaff (2025). Zenodo.  
  DOI: \href{https://doi.org/10.5281/zenodo.17209646}{10.5281/zenodo.17209646}.
\end{itemize}

Together, these works define the axioms, kinematic rules, and stiffness interpretation of mass in the Planck-Cell medium, providing the foundation on which the present gravitational model is constructed.

% ----------------
% Bibliography (BibTeX by default)
% ----------------
\clearpage
\nocite{*} % avoid Empty thebibliography even with zero explicit \cite
\bibliographystyle{plain}
\bibliography{bibliography}

\end{document}
